\documentclass{article}
\usepackage[utf8]{inputenc}

\title{PS1 -JCJ}
\author{joe.castiglione.jr }
\date{January 2020}

\usepackage{natbib}
\usepackage{graphicx}

\begin{document}

\maketitle

\section{Introduction}
Quite frankly if you were to ask me a year ago if I would attacking my second coding class of my graduate career I would say you're crazy. My coding interest stems from the application it has to apply towards sports performance. My hope it to be able to create my own measurements for strength and conditioning and provide different values for evaluation. Professor Wang's last class is what created a desire to continue. There was so many times that I did not know what was going on but he provided such detail that somehow it all worked out. 
I do not know exactly what I would want to do for this class' project. But maybe something of taking the vertical jump of some of the University's athletes and running the comparisons and breaking them down by position. (The last semester I interned with the Strength & Conditioning department). Then using different ways to find tendencies to help athletes know when they are at their best. I want to be able to combine this class' project with my research method class to have something that will provide me great show in the future.  Some of my goals in this class are to build the skills and knowledge to know what to use when breaking down data. I know that seems vague but it seemed last semester I was able to use different packages and understand what they said but never understood why we should use them or when. 
\section{Equation}
a2 + b2 = c2


\section{Conclusion}
``I am not for sure what to do with my hands'' - Ricky Bobby 
\citep{}

\bibliographystyle{plain}
\bibliography{references}
\end{document}
